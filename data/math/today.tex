
\documentclass[12pt]{article}
\usepackage{amsmath, amssymb, amsthm}
\usepackage{geometry}
\geometry{margin=1in}

\begin{document}

\section*{Advanced Math Exercises for CS/AI Students}

\subsection*{A. Computational Problem}
Given a matrix \( A \in \mathbb{R}^{n \times n} \) with eigenvalues \( \lambda_1, \lambda_2, \ldots, \lambda_n \), prove that the determinant of \( A \) can be expressed as \( \det(A) = \prod_{i=1}^{n} \lambda_i \). Furthermore, if \( A \) is invertible, show that the eigenvalues of \( A^{-1} \) are given by \( \frac{1}{\lambda_i} \) for each eigenvalue \( \lambda_i \) of \( A \).

\subsection*{B. Proof Problem}
Prove that for any two vectors \( \mathbf{u}, \mathbf{v} \in \mathbb{R}^n \), the following inequality holds: \( \| \mathbf{u} + \mathbf{v} \|^2 \leq 2(\| \mathbf{u} \|^2 + \| \mathbf{v} \|^2) \). Use the properties of the dot product and the Cauchy-Schwarz inequality in your proof.

\end{document}
